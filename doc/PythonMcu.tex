\input{include/settings.sty}
\input{include/hyphenation.sty}

\title{Python MCU}
\author{Martin Zuther}

\begin{document}

\title{Python MCU}

\subtitle{
  \normalsize{\textrm{\textmd{
        \vfill
        Mackie Host Controller written in Python
        \vfill
        \vspace{10em}
        \vfill
      }}}
}

\author{\normalsize\copyright{} 2011
  \href{http://www.mzuther.de/}{Martin Zuther}}

\date{\normalsize \emph{Last edited on \today}}

\maketitle

\tableofcontents

\clearpage  % layout

\chapter{Mackie Control}
\label{chap:mackie_control}

\textbf{Mackie Control} (abbreviated to MCU), its predecessor
\textbf{Logic Control} and their respective extension units
(\textbf{XT}) are control surfaces that use a proprietary MIDI
protocol to control digital audio workstations (DAWs), especially
their mixing sections.  However, these control surfaces have two
drawbacks: they are definitely not cheap and require quite a bit of
space.  If these are no restrictions to you, by all means go and try
them.  But don't stop reading here, as this application might be
useful for Mackie Control users as well.

If, on the other hand, you have a certain lack of money or available
space (or you have a controller that is simply too good to be
exchanged for an MCU), you might have found just the application you
need.  It might not support your MIDI hardware controller yet, but if
you know a bit of \textbf{Python} (or a programmer) it's pretty easy
to change that.


\chapter{Installation}
\label{chap:python_mcu}

Download and install the latest version of
\href{http://www.python.org/}{Python 2.6} on your computer.  As of
August 2011, there is no installer of Python 2.6.7 for Microsoft
Windows yet.  Users of this operating system may download
\href{http://www.python.org/download/releases/2.6.6/}{Python 2.6.6},
instead.

Please also download and install these libraries:

\begin{compactitem}
\item \href{http://www.pyside.org//}{PySide} (version 1.0.5 or above)
\item \href{http://www.pygame.org/}{pygame} (version 1.9.1 or above;
  please note that pygame's MIDI implementation is still in its
  infancy and may occasionally crash \application{Python MCU})
\end{compactitem}

You'll also need virtual MIDI ports or cables to connect
\application{Python MCU} to your DAW and hardware controller.  I have
successfully used \application{MidiYoke NT} (Microsoft Windows) and
\application{JACK} (GNU/Linux), but others may work just as well.

When you're done, simply run the file \path{PythonMcu.py} in the
directory \path{src} by either double-clicking it or starting a
console window, moving to the directory \path{src} and running
\texttt{python PythonMcu.py}.


\chapter{Tested configurations}
\label{chap:tested_configurations}

\section{Hardware controllers}

\begin{compactitem}
\item Novation ZeRO SL MkII
\end{compactitem}

\section{Microsoft Windows XP}

\begin{compactitem}
\item Ableton Live 8
\item Cockos Reaper 4
\item Emagic Logic 5
\end{compactitem}

\section{Apple Mac}

I haven't got a Mac, but things should work just as well.  Please
report working configurations to me.  Thanks!

\section{GNU/Linux}

\begin{compactitem}
\item ardour 2 \emph{(skipped a lot of commands, though)}
\end{compactitem}

\chapter{Extending Python MCU}
\label{chap:extending_python_mcu}

\application{Python MCU} consists of three parts:

\begin{itemize}

\item the \textbf{MackieHostControl} class communicates with a
  connected sequencer using the Mackie Control protocol, translating
  the raw protocol to something much easier to read and use

\item the generalised \textbf{MidiControllerTemplate} class and its
  more specific relatives handle all the details of sending data to
  and receiving data from your hardware controller, again translating
  raw protocols to something easier to read and use

\item finally, the \textbf{McuInterconnector} class connects the
  \textbf{MackieHostControl} and \textbf{MidiControllerTemplate}
  classes (and thus your DAW and hardware controller) with each other

\end{itemize}

This modular design means that the application happily works away with
the irrelevant implementation details being effectively hidden from
you.  As long as you adhere to the internal protocol, you may easily
extend \application{Python MCU} by deriving a class from
\textbf{MidiControllerTemplate} to include your own controller.

If all this means nothing to you, go find yourself a Python programmer
(or learn Python yourself, it's rather easy and a lot of fun!).  As
long as you know the relevant MIDI implementation of your hardware
controller and worked out a suitable layout of its buttons and
controllers, hooking into \application{Python MCU} is pretty simple.
If you don't believe me, just have a look into the \path{src/Hardware}
directory.

\input{include/gpl_v3.tex}

\end{document}


%%% Local Variables:
%%% mode: latex
%%% mode: outline-minor
%%% TeX-command-default: "Rubber"
%%% TeX-PDF-mode: t
%%% ispell-local-dictionary: "british"
%%% End:
